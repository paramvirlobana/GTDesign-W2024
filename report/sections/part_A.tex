\chapter{Cycle Calculations} \label{sec:cyclecalcs}
The objective of this technical report is to present a design for a new family of gas turbine engines under a risk-sharing partnership with Pratt \& Whitney Canada.
The turboshaft engine design should have a three shaft arrangement for sea level operation at $74^{o}F$.
The engine features a single-stage low-pressure compressor (LPC) and axial turbine 
(LPT), followed by a centrifugal high-pressure compressor (HPC) and axial 
high-pressure turbine (HPT). The power turbine (PT) drives the helicopter rotor 
through a reduction gearbox. 

The following figure presents the engine schematics along with all the stations.
\vspace{20pt}
\begin{figure}[H]
  \centering
  \begin{tikzpicture}[scale=0.5]

%\draw [lightgray](0,0) grid (30, 20);

% Inlet 
\draw [thick] (1, 5) -- (4.5, 5);
\draw [thick] (1, 8) -- (3, 8);
\draw [thick] (3, 8) arc (270:360:1);
\draw [thick] (4, 9) -- (4, 14);
\draw [thick] (4, 14) arc (180:120:1);

% SHAFTS
\fill [lightgray] (26, 9.75) rectangle (3, 10.25); % Power turbine shaft
\fill [darkgray] (22, 9.75) rectangle (5, 10.25); % High Pressure shaft
\fill [red] (8.5, 9.75) rectangle (18, 10.25); % Low Pressure shaft
%==============================================================================================

% Internal Engine Components
\draw [thick] (1, 7) -- (5, 7);
\draw [-{Stealth[scale=3]}] (1, 7) -- (3, 7);
%LPC
\draw [thick] (5,5) -- (5, 15) -- (7.5, 13.5) -- (7.5, 6.5) -- (5,5);
%---------------------
\draw [thick] (7.5, 13.5) -- (7.5, 15) -- (8.5, 15) -- (8.5, 13.5) ;
\draw [-{Stealth[scale=2]}] (7.5, 15) -- (8.25, 15);
%---------------------
%HPC
\draw [thick] (8.5,6.5) -- (8.5, 13.5) -- (11, 12) -- (11, 8) -- (8.5,6.5);

% Combustion Chamber
\draw [thick] (11, 12) -- (11, 14.5) -- (12, 14.5);
\draw [-{Stealth[scale=3]}] (11, 12) -- (11, 13.5);
\draw [thick] (13, 14.5) circle [radius=1];
\draw [thick] (14, 14.5) -- (15, 14.5) -- (15, 12);
\draw [-{Stealth[scale=3]}] (15, 14.5) -- (15, 13);

%HPT
\draw [thick]  (15, 13) -- (15, 7) -- (16.5, 8) -- (18, 8) -- (18, 12) -- (16.5, 12) -- (15, 13);
\draw [thick]  (16.5, 12) -- (16.5, 7);
%---------------------
\draw [thick] (18, 12) -- (18, 15) -- (19, 15) -- (19, 12) ;
\draw [-{Stealth[scale=2]}] (18, 15) -- (18.75, 15);
%---------------------


%LPT
\draw [thick] (22, 6.5) -- (22, 13.5) -- (19, 12) -- (19, 8) -- (22, 6.5);

%---------------------
\draw [thick] (22, 12) -- (22, 15.5) -- (23, 15.5) -- (23, 14) ;
\draw [-{Stealth[scale=2]}] (22, 15.5) -- (22.75, 15.5);
%---------------------

% Power Turbine
\draw [thick] (26,4.5) -- (26, 15.5) -- (23, 14) -- (23, 6) -- (26, 4.5);
\draw [thick] (26, 8.25) -- (29.5, 8.25);
\draw [-{Stealth[scale=3]}] (26, 8.25) -- (28, 8.25);

% Exhaust
\draw [thick] (27, 7) -- (29, 8);
\draw [thick] (27, 13) -- (29, 12);

%==============================================================================================

% Gearbox
\draw [thick] (1, 9) rectangle (3, 11);
\coordinate[label=above: \small{Gearbox}] (Y) at (2, 11);
% Naming


% Compressor
\coordinate[label=above: LPC] (A) at (6.25,10.5);
\coordinate[label=above: HPC] (B) at (9.75,10.5);

% Compressor
\coordinate[label=above: CC] (C) at (13,14.15);

% Turbine
\coordinate[label=above: HPT] (D) at (16.5,12.5);
\coordinate[label=above: \tiny{Stator}] (DA) at (15.75,10.5);
\coordinate[label=above: \tiny{Rotor}] (DB) at (17.25,10.5);



\coordinate[label=above: LPT] (E) at (20.5,10.5);
\coordinate[label=above: PT] (G) at (24.5,10.5);


\coordinate[label=above: Inlet] (H) at (2.5,5.5);
\coordinate[label=above: \rotatebox{90} {Exhaust}] (H) at (28,8.5);

% Stations
\coordinate[label=below: 1] (Y) at (4.75, 5);

\coordinate[label=above: 2] (Y) at (8, 15.25);

\coordinate[label=above: 3] (Y) at (11, 14.5);

\coordinate[label=above: 4] (Y) at (15, 14.5);

\coordinate[label=above: 5] (Y) at (18, 15.25);

\coordinate[label=above: 6] (Y) at (22, 15.75);

\coordinate[label=below: 7] (Y) at (26.5, 7);

\coordinate[label=below: 0] (Y) at (0.5, 5);

\coordinate[label=below: 0] (Y) at (29.5, 8.25);



% Cooling

% HPT VANE
\draw [blue, thick] (11, 8) -- (11, 4) -- (15, 4);
\draw [blue, thick, ->] (15, 4) -- (15, 7);
%HPT DISC
\draw [blue, thick] (15, 4) -- (16.5, 4);
\draw [blue, thick, ->] (16.5, 4) -- (16.5, 8);
% LPT DISC
\draw [blue, thick] (16.5, 4) -- (20.5, 4);
\draw [blue, thick, ->] (20.5, 4) -- (20.5, 7.25);
% PT DISC
\draw [blue, thick] (20.5, 4) -- (24.5, 4);
\draw [blue, thick, ->] (24.5, 4) -- (24.5, 5.25);

\end{tikzpicture}

  \caption{Engine Schematic}
  \label{fig:enter-label}
\end{figure} 

\clearpage

The key engine requirements are summarized in \autoref{tab:enginecharacteristics}.
\begin{table}[H]
    \caption{Engine Characteristics}
    \label{tab:enginecharacteristics}
    \centering
    \begin{tabular}[H]{l l l l}
    \toprule[1pt]
    \multicolumn{2}{c}{\textit{Engine Inlet}}    & \multicolumn{2}{c}{\textit{Compressor}}                    \\
    \midrule
    Inlet Loss     & 1.00 \%              &   Mass Flow                          &  5.21 $kg/sec$  \\
    && LPC PR & 4 \\
    && LPC Target Efficiency & 86.5 \% \\
    && HPC PR & 3 \\
    && HPC Target Efficiency & 85.5 \% \\
    && HPC Bleed Air (exit) & 9 \% \\
    \midrule
    \multicolumn{2}{c}{\textit{Combustor}}    & \multicolumn{2}{c}{\textit{Turbine}}                    \\
    \midrule
    Fuel to Air ratio & 0.02 & HPT Target Efficiency& 83 \%-85 \%\\
    Heating Value & 40007.2 $kJ/kg$ & HPT Vane Cooling Air& 3 \%\\
    Efficiency & 0.99 & HPT Disk Cooling Air& 1.65 \%\\
    Pressure Loss & 1.8 \% &LPT Target Efficiency& 91 \% \\
    RTDF & 5 \% &LPT Disk Cooling Air& 1.1 \%\\
    && ITD loss & 0.6 \% \\
    && PT Target Efficiency & 92 \%\\
    && PT disk cooling air & 1.25 \% \\
    && Exhaust Loss & 2.0 \% \\
    && Exhaust Mach Number & 0.15 \\
    \midrule[1pt]
    \end{tabular}
\end{table} 


This section discusses the method used for the cycle calculations. The calculations for each stage are presented as well.

\section{Inlet}
Since $P_{00} = P_a$ and $T_{00} = T_a$, the pressure drop across the inlet can be calculated as:
$$P_{01} = 0.99 \times P_{00} = 100.311 \; kPa$$
$$T_{01} = 296.483 \; K$$

\section{Low Pressure Compressor} \label{lpc}
The pressure at the LPT exit $(P_{02})$ can be calculated as:
$$P_{02} = PR_{LPC} \times P_{01} = 401.247 \; kPa$$

The temperature can be obtained using the following relation:
\begin{equation}
  T_{02} - T_{01} = \frac{T_{01}}{\eta_{lpc}} \left[  \left( \frac{P_{02}}{P_{01}} \right) ^{(\gamma - 1)/\gamma}  - 1  \right]
\end{equation}

$$T_{02} = 463.059 \; K$$

Therefore, the specific work done for the LPC:
$$W_{LPC} = c_{p, \, air} \times (T_{02} - T_{01}) = 167.409 \; kJ/kg$$

\section{High Pressure Compressor}

Following the steps from Section \ref{lpc}, $P_{03}$, $T_{03}$ and $W_{HPC}$ can be calculated:

$$P_{03} = PR_{HPC} \times P_{02} = 1203.741 \; kPa$$

$$T_{03} = 662.764 \; K$$

$$W_{HPC} = c_{p, \, air} \times (T_{03} - T_{02}) = 200.703 \; kJ/kg$$

Since the bleed air is obtained from the HPC exit, the resulting mass flow rate entering the combustion chamber:
$$\dot{m}_{combustion} = 0.91 \times \dot{m}_{total} = 4.741 \; kg/s$$

\section{Combustion Chamber}
Pressure at the exit of the combustion chamber can be calculated using the combustion chamber pressure loss:
$$P_{04} = P_{03} \times 0.982 = 1182.073 \; kPa$$

The temperature at the exit of the combustion chamber can be calculated using the following relation:
$$T_{04} = \frac{c_{p, \, air} \times T_{03} + f \times Q \times \eta_{cc}}{(1 + f) \times c_{p, \, gas}} = 1245.320 \; K$$
\clearpage
\section{High Pressure Turbine} \label{hpt_calcs}

The following points are noted for the calculations of the turbine temperatures and pressures with cooling \cite{saravanamuttoo2017}:
\begin{itemize}
  \item Turbine air cooling percentage considered as the percent flow of turbine inlet flow.
  \item Stator bleed contributes to power developed.
  \item Disc bleed do not contribute to power developed.
\end{itemize}

The combusted gas entering the high pressure turbine also included the mass of the fuel. This is calculated as:
$$\dot{m}_{HPT} = \dot{m}_{combustion} + 0.02 \times \dot{m}_{combustion} = 4.835 \; kg/s$$
Similarly, the mass flow of air to be used for vane and disc cooling for the HPT is as follows:
$$\dot{m}_{vane,\, HPT} = \dot{m}_{turbine} \times 0.03 = 0.1450 \; kg/s$$
$$\dot{m}_{disc,\, HPT} = \dot{m}_{turbine} \times 0.0165 = 0.0797 \; kg/s$$

To obtain the temperature after the vane, the following energy balance can be performed:
$$T_{\text{hpt after vane}} = \frac{{(\dot{m}_{\text{turbine}} \cdot c_{p_{\text{gas}}} \cdot T_{04}) + (\dot{m}_{vane,\, HPT} \cdot c_{p_{\text{air}}} \cdot T_{03})}}{{c_{p_{\text{gas}}} \cdot (\dot{m}_{\text{turbine}} + \dot{m}_{vane,\, HPT})}}$$
$$T_{\text{hpt after vane}} =  1225.9485\; K$$

The temperature after the rotor can be evaluated based on the power required by the HPC:
$$T_{\text{hpt after rotor}} = T_{04} - \frac{{1.01 \cdot W_{\text{hpc}}}}{{(m_{\text{turbine}} + \dot{m}_{vane,\, HPT}) \cdot c_{p,\,gas}}} = 1041.2535\; K$$

The HPT exit temperature can be calculated by accounting for the disc bleed air:
$$T_{05} = \frac{{((\dot{m}_{\text{turbine}} + \dot{m}_{vane,\, HPT}) \cdot c_{p_{\text{gas}}} \cdot T_{\text{hpt after rotor}}) + 
(\dot{m}_{disc,\, HPT} \cdot c_{p_{\text{air}}} \cdot T_{03})}}{{c_{p_{\text{gas}}} \cdot (\dot{m}_{\text{turbine}} + \dot{m}_{vane,\, HPT} + 
\dot{m}_{disc,\, HPT})}}
$$


$$T_{05} = 1033.9843\; K$$

The pressure can be evaluated using the isentropic efficiency. For this design, the HPT efficient has been selected as 84 \%. The following relation is used:
\begin{equation} \label{eq:pressure_eq}
  T_{04} - T_{05} = \eta_{HPT} T_{04} \left[ 1 - \left(  \frac{1}{P_{04}/P_{05}} \right) ^{(\gamma - 1)/\gamma}\right]
\end{equation}

$$P_{05} = 517.9223\; kPa$$

\section{Low Pressure Turbine}
The total mass of the gas entering the LPT includes the cooling air from the HPT as well:
$$\dot{m}_{LPT} = \dot{m}_{HPT} + \dot{m}_{disc,\, HPT}$$
$$\dot{m}_{disc,\, HPT} = 0.011 \times \dot{m}_{turbine}$$
Following similar calculation procedure as seen in Section \ref{hpt_calcs}, the following values are calculated:
$$T_{\text{lpt after rotor}} = 882.3564 \; K$$
$$T_{06} = 879.2135 \; K$$

\section{Exhaust}
Since the exit conditions are already given in the problem statement, the calculations for the exhaust are carried out first.
Using the exit loss and the exit mach number, the exit total pressure can be calculated using the following relation:
\begin{equation}
  \frac{P_{08}}{P_0} = \left[  1 + \left(  \frac{\gamma - 1}{2} \right) M_{exit}^2  \right]^{\frac{\gamma}{\gamma-1}}   
\end{equation}

$$P_{08} = 102.853\; kPa$$

Therefore, the Power Turbine exit pressure can be calculated as:
$$P_{07} = 1.02 \times P_{08} = 104.910\; kPa$$

\section{Power Turbine}
The total mass of the gas entering the PT includes the cooling air from the HPT and LPT:
$$\dot{m}_{PT} = \dot{m}_{LPT} + \dot{m}_{disc,\, LPT}$$

Incorporating ITD losses:
$$P_{06, \, PT} = (1 - ITD_{loss}) \times P_{06} = 240.755 \; kPa$$

Following similar calculation procedure from Section \ref{hpt_calcs}:
$$T_{\text{pt after rotor}} = 727.5310 \; K$$
After adding disc cooling air:
$$T_{07} = 725.8100 \; K$$


\section{Work and SFC}
The work for the HPT can be computed as following.

$$W_{HPT} = c_{p, \, gas} (T_{06} - T_{07}) = 212.0 kJ/kg$$


Work can be calculated as follows:
$$W_{PT} = c_{p, \, gas} (T_{06} - T_{07}) \times 0.99 = 172.4 \; kJ/kg$$

$$SFC = \frac{3600 \times 0.02}{W_{PT}} = 0.4176$$


\clearpage