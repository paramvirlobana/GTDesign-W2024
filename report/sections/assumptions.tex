\section{Assumptions} \label{sec:assumptions}
The following assumptions are made \cite{saravanamuttoo2017}
\begin{itemize}
  \item A low value of $\psi$ leads to having more stages for a given turbine output. Hence, the objective will be to keep the value of $\psi$ from 3 to 5 based on current aircraft practices \cite{gt_theory}.
  \item A low value of $\phi$ leads to a larger annulus area for a given mass flow. A value of $\phi$ from 0.8 to 1.0 is recommended from literature \cite{gt_theory}.
  \item As an initial guess, a 50 per cent reaction design point is chosen while eliminating all values near 0 \cite{axialflowturbines}.
  \item Additionally, for jet propulsion, the swirl angle at the exit stage must be kept low since it leads to losses on the jet pipe and propelling nozzle \cite{farokhi2021aircraft}.
\end{itemize} 

\autoref{tab:designconstraints} summarizes some hard and soft design constraints that will be used during the
design process.
\begin{table}[H]
    \caption{Summary of Design Constraints}
    \label{tab:designconstraints}
    \centering
    \begin{tabular}[H]{l l l l}
    \toprule[1pt]
    \multicolumn{2}{c}{\textit{Hard Design Constraints}}    & \multicolumn{2}{c}{\textit{Soft Design Constraints}}                    \\
    \midrule
    $M_2 > 1$     & (choked first nozzle)         &   $C_a = 0$           &  (acceptable) \\
    $M_{3r} < 1$     & (unchoked rotor exit flow)       &  $r_m \neq constant$     & (pitchline radius is variable)  \\
    $\Lambda > 0$ & (positive degree of reaction) &  $\alpha_{exit} \neq 0$  & (turbine exit swirl not zero) \\
    Stages $<$ 2  & (to keep the cost and weight low)        &  $0.4 < \Lambda < 0.5$     &      (wide range of $\Lambda$)\\
                  &                               &   $3 < \psi < 5$         &  (wide range of $\psi$)    \\
                  &                               &     $0.8 < \phi < 1.0$   &     (wide range of $\phi$)   \\
                  &                               & $40^{\circ} < \alpha_2 < 70^{\circ}$ & (nozzle exit angle range) \\
    \midrule[1pt]
    \end{tabular}
  \end{table}

\autoref{tab:additional_information} provides some values that have been chosen to solve the problem.
\begin{table}[H]
  \caption{Additional Information}
  \label{tab:additional_information}
  \centering
  \begin{tabular}[H]{l c l}
  \toprule[1pt]
  \textbf{Parameter} & \textbf{Value} & \textbf{Units}  \\
  \midrule
  R                 &   287  &  [J/kg .K]  \\
  $c_{p, \; turbine}$   &   1243.67 &  [J/kg .K]  \\
  $\gamma_{turbine }$ & 1.30& [/]\\

  
  \midrule[1pt]
  \end{tabular}
\end{table}

\clearpage